\section{Conclusion}
To determine the useful life of a 9V battery, battery voltage was connected to Raspberry Pi, and Raspberry Pi collected data for fifteen hours, and as it reached the voltage value considered “dead”, it stopped collecting data. Ohm’s law was used to get an idea of what resistance value should be used for the voltage divider. Also, the relationship between $V_{out}$ and $V_{in}$ in a voltage divider was used to find $V_{battery}$, python program program was measuring the voltage that goes in our Raspberry Pi at a that time and it was multiplied by division factor to calculate an actual voltage of the battery. It was calculated that the capacity of the battery is $Capacity = 394.65 \ mA*h$, which is expected by looking at continuous discharge in the datasheet of the battery.
